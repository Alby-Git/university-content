% ----------------------- TODO ---------------------------
% Diese Daten müssen pro Blatt angepasst werden:
\newcommand{\NUMBER}{1}
\newcommand{\EXERCISES}{2}
\newcommand{\AUFGABENSTART}{1}
\newcommand{\DEADLINE}{Wednesday, 2024-04-24, 15:59}
% Diese Daten müssen einmalig pro Vorlesung angepasst werden:
\newcommand{\COURSE}{Software Engineering}
\newcommand{\TUTOR}{}
\newcommand{\STUDENTA}{Albert Ratschinski (5154309)}
% ----------------------- TODO ---------------------------

\documentclass[a4paper]{scrartcl}

\usepackage[utf8]{inputenc}
\usepackage[ngerman]{babel}
\usepackage{dsfont}
\usepackage{amsmath}
\usepackage{amssymb} % Added for square symbols
\usepackage{fancyhdr}
\usepackage{color}
\usepackage{graphicx}
\usepackage{lastpage}
\usepackage{listings}
\usepackage{tikz}
\usepackage{pdflscape}
\usepackage{subfigure}
\usepackage{float}
\usepackage{polynom}
\usepackage{hyperref}
\usepackage{tabularx}
\usepackage{forloop}
\usepackage{geometry}
\usepackage{fancybox}
\usepackage{stmaryrd}
\usepackage{bbold}

\usetikzlibrary{calc,arrows}
\usepackage{listings}
\lstset{
    basicstyle=\small\ttfamily,
    breaklines=true,
    numbers=left,
    numberstyle=\tiny,
    frame=tb,
    columns=fullflexible
}

%Größe der Ränder setzen
\geometry{a4paper,left=3cm, right=3cm, top=3cm, bottom=3cm}

%Kopf- und Fußzeile
\pagestyle {fancy}
\fancyhead[L]{Tutor: \TUTOR}
\fancyhead[C]{\COURSE}
\fancyhead[R]{\today}

\fancyfoot[L]{}
\fancyfoot[C]{}
\fancyfoot[R]{Seite \thepage /\pageref*{LastPage}}

\newcounter{aufgabe}

%Formatierung der Überschrift, hier nichts ändern
\def\header#1#2{
  \begin{center}
    {\Large Exercise Sheet #1}\\
    {(Deadline #2)}
  \end{center}
}

\newcommand{\nextAufgabe}{
  \section*{Aufgabe \theaufgabe}
  \stepcounter{aufgabe}
}
\newcommand{\circled}[2]{
  \tikz[baseline=(char.base)]{
  \node[shape=circle,draw,color=#2,inner sep=2pt] (char) {#1};}
}

\newcommand{\rc}[1]{
  \circled{#1}{red}
}
\newcommand{\gc}[1]{
  \circled{#1}{green}
}
\newcommand{\bc}[1]{
  \circled{#1}{blue}
}

\newcommand{\R}{
  \mathbb{R}
}
\newcommand{\N}{
  \mathbb{N}
}
\newcommand{\Z}{
  \mathbb{Z}
}
\newcommand{\Q}{
  \mathbb{Q}
}
\newcommand{\C}{
  \mathbb{C}
}

\newcounter{spalte}
\newcounter{zeile}

\newcommand{\spalte}[3]{
\left(\begin{array}{c}
  #1   \\
  #2   \\
  #3   \\
\end{array}\right)
}
\newcommand{\spaltef}[4]{
\left(\begin{array}{c}
  #1   \\
  #2   \\
  #3   \\
  #4   \\
\end{array}\right)
}
\newcommand{\kopf}[4]{
  
    idx & 
    %x werte
      \setsepchar{ }
      \forloop{spalte}{0}{\value{spalte} < #1}%
        {
        \readlist\arg{#3}
        \arg[\fpeval{\thespalte+1}] &
        } 
      %funktionen
      \setsepchar{ }
      \forloop{spalte}{0}{\value{spalte} < #2}%
        {
        \readlist\arg{#4}
        \arg[\fpeval{\thespalte+1}] \ifthenelse{\fpeval{#2-1}=\thespalte}{}{&}
        }
        \\\hline 
}

\begin{document}

\begin{tabularx}{\linewidth}{m{0.5 \linewidth} X}
  \begin{minipage}{\linewidth}
    \STUDENTA\\
  \end{minipage} &
\end{tabularx}
\setcounter{aufgabe}{\AUFGABENSTART}%
\header{Nr. \NUMBER}{\DEADLINE}

% ----------------------- TODO ---------------------------

\begin{center}
  \begin{tabular}{|c|ccc|ccc|}
    \hline
    Task      & 1.1         & 1.2         & 1.3         & 2.1       & 2.2       & 2.3       \\
    \hline
    Completed & $\boxtimes$ & $\boxtimes$ & $\boxtimes$ & $\square$ & $\square$ & $\square$ \\
    \hline
    Feedback  & $\square$   & $\square$   & $\square$   & $\square$ & $\square$ & $\square$ \\
    \hline
  \end{tabular}
\end{center}


\section*{Exercise 1 - Does Software Development Success Matter?}

\subsection*{Exercise 1.1}
\begin{itemize}
  \item \textbf{G. Volkswagen Diesel Emissions Scandal (2015)}

  \item \textbf{General Description:} The Volkswagen diesel emissions scandal, also known as ''Dieselgate,'' emerged in 2015 and involved the deliberate manipulation of emissions test results by Volkswagen (VW) to make their diesel-powered vehicles appear cleaner and more environmentally friendly than they actually were.

  \item \textbf{Software-Related Issue and Consequences:}
        \begin{itemize}
          \item The software-related issue at the heart of the scandal involved the installation of illegal software, known as a ''defeat device,''
                in millions of VW diesel vehicles worldwide. This software was designed to detect when the vehicle was undergoing emissions testing
                and to activate emission control systems to reduce pollutants.
                However, during normal driving conditions, the emission controls were deactivated, allowing the vehicles to
                emit nitrogen oxide (NOx) pollutants at levels far exceeding regulatory limits.

          \item The consequences of this deception were significant both for VW and the environment.
                While there was no immediate loss of life directly attributable to the scandal,
                the increased levels of NOx emissions from affected vehicles posed serious health risks to populations exposed to air pollution,
                contributing to respiratory illnesses and environmental damage.

          \item Volkswagen faced a barrage of lawsuits, regulatory fines, and reputational damage as a result of the scandal.
                The company was forced to recall millions of vehicles, pay billions of dollars in settlements and fines,
                and implement extensive remediation efforts to address the environmental impact of their actions.
        \end{itemize}

  \item \textbf{Quantification of Damages and Relevance:}
        \begin{itemize}
          \item The financial repercussions of the Volkswagen diesel emissions scandal were substantial.
                Volkswagen ultimately paid over \$30 billion in fines, penalties, settlements, and vehicle buyback programs related to the scandal.
                Additionally, the company's market value plummeted, and its reputation suffered irreparable harm, leading to a loss of consumer trust and confidence in the brand.

          \item The scandal also had far-reaching implications for the automotive industry as a whole, prompting increased scrutiny of emissions testing procedures,
                regulatory oversight, and corporate ethics.
                It served as a stark reminder of the importance of compliance with environmental regulations and the potential consequences of unethical behavior in business.
        \end{itemize}

  \item \textbf{Sources}
        \begin{itemize}
          \item 'https://de.wikipedia.org/wiki/Abgasskandal'
          \item 'https://www.tagesschau.de/wirtschaft/verbraucher/abgassskandel-vw-verfahren-myright-100.html'
        \end{itemize}
\end{itemize}

\subsection*{Exercise 1.2}
\textbf{Developer (Volkswagen):}
\begin{itemize}
  \item \textbf{Success Perspective:} Initially successful in achieving deceptive objectives and gaining market advantage.
  \item \textbf{Failure Perspective:} Led to severe financial and reputational damage, loss of trust, and legal consequences.
\end{itemize}

\textbf{Customer (Consumers and Regulatory Authorities):}
\begin{itemize}
  \item \textbf{Success Perspective:} None, as consumers were deceived, and regulatory standards were circumvented.
  \item \textbf{Failure Perspective:} Resulted in financial losses, environmental harm, and undermined trust in both Volkswagen and regulatory authorities.
\end{itemize}

\textbf{User (Drivers and the General Public):}
\begin{itemize}
  \item \textbf{Success Perspective:} None, as drivers were misled by deceptive marketing and false claims.
  \item \textbf{Failure Perspective:} Increased emissions contributed to environmental pollution and health risks, eroded public trust in corporate ethics.
\end{itemize}
In summary, while the software development initially appeared successful for Volkswagen in achieving deceptive objectives, it ultimately resulted in significant
failure from the perspectives of developers, customers, and users due to severe financial, reputational, and environmental consequences of the scandal.

\newpage

\subsection*{Exercise 1.3}
\textbf{Therac-25:}
\begin{itemize}
  \item \textbf{Requirements:} Flaws in requirements specification and validation.
  \item \textbf{Design:} Lack of error handling and fault tolerance mechanisms.
  \item \textbf{Quality Assurance:} Inadequate testing and validation.
  \item \textbf{Management:} Poor communication and risk management.
\end{itemize}

\textbf{Ariane-5 Rocket:}
\begin{itemize}
  \item \textbf{Requirements:} Failure to consider new rocket's performance.
  \item \textbf{Design:} Oversight in data conversion and error handling.
  \item \textbf{Quality Assurance:} Lack of testing under realistic conditions.
  \item \textbf{Management:} Coordination and risk management issues.
\end{itemize}

\textbf{Toll Collect:}
\begin{itemize}
  \item \textbf{Requirements:} Ambiguous requirements and poor integration.
  \item \textbf{Design:} Lack of standardized interfaces and modularization.
  \item \textbf{Quality Assurance:} Insufficient testing and regression testing.
  \item \textbf{Management:} Poor project planning and resource allocation.
\end{itemize}

\textbf{Volkswagen Diesel Scandal:}
\begin{itemize}
  \item \textbf{Requirements:} Intentional manipulation of emissions tests.
  \item \textbf{Design:} Implementation of illegal defeat devices.
  \item \textbf{Quality Assurance:} Compromised testing and lack of transparency.
  \item \textbf{Management:} Lack of accountability and governance.
\end{itemize}

\section*{Exercise 2 - Lines of Code Metrics}
\subsection*{Exercise 2.1}

\begin{itemize}
  \item $LOC_{tot} = 73$
  \item $LOC_{ne} = 64$
  \item $LOC_{pars} = 48$
\end{itemize}

\subsection*{Exercise 2.2}
\subsubsection*{(a)}
\textbf{Program 1:}

\begin{lstlisting}[language=Java]
public class Program1 {
    public static void main(String[] args) {
        // This program calculates the factorial of 5
        
        int n = 5;
        int factorial = 1;
        for (int i = 1; i <= n; i++) {
            factorial *= i;
        }
        System.out.println("Factorial of " + n + ": " + factorial);
    }
}
\end{lstlisting}

\textbf{Program 2:}

\begin{lstlisting}[language=Java]
public class Program2 {
    public static void main(String[] args) {
        // This program calculates the factorial of 5
        
        int n = 5;
        int factorial = 1;
        factorial = factorial * 1 * 2 * 3 * 4 * 5;
        System.out.println("Factorial of " + n + ": " + factorial);
    }
}
\end{lstlisting}

\subsubsection*{}
Both programs calculate the factorial of 5.
Program 1 uses a loop to calculate the factorial incrementally,
while Program 2 directly computes the factorial using the multiplication of individual numbers.
As a result, Program 1 has substantially more lines of code than Program 2, leading to different $LOC_{pars}$ values.

\subsubsection*{(b)}
No, the example provided is not a rare exception.
It's common for semantically equivalent programs to have substantially different metric values, such as $LOC_{pars}$,
due to variations in coding style, algorithm choice, and implementation details.


\end{document}
