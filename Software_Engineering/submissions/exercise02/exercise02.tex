% ----------------------- TODO ---------------------------
% Diese Daten müssen pro Blatt angepasst werden:
\newcommand{\NUMBER}{1}
\newcommand{\EXERCISES}{2}
\newcommand{\AUFGABENSTART}{1}
\newcommand{\DEADLINE}{Wednesday, 01-05-2024, 15:59}
% Diese Daten müssen einmalig pro Vorlesung angepasst werden:
\newcommand{\COURSE}{Software Engineering}
\newcommand{\TUTOR}{}
\newcommand{\STUDENTA}{Albert Ratschinski (5154309)}
\newcommand{\STUDENTB}{Severin Plewe (5333060)}
% ----------------------- TODO ---------------------------

\documentclass[a4paper]{scrartcl}

\usepackage[utf8]{inputenc}
\usepackage[ngerman]{babel}
\usepackage{dsfont}
\usepackage{amsmath}
\usepackage{amssymb} % Added for square symbols
\usepackage{fancyhdr}
\usepackage{color}
\usepackage{graphicx}
\usepackage{lastpage}
\usepackage{listings}
\usepackage{tikz}
\usepackage{pdflscape}
\usepackage{subfigure}
\usepackage{float}
\usepackage{polynom}
\usepackage{hyperref}
\usepackage{tabularx}
\usepackage{forloop}
\usepackage{geometry}
\usepackage{fancybox}
\usepackage{stmaryrd}
\usepackage{bbold}

\usetikzlibrary{calc,arrows}
\usepackage{listings}
\lstset{
    basicstyle=\small\ttfamily,
    breaklines=true,
    numbers=left,
    numberstyle=\tiny,
    frame=tb,
    columns=fullflexible
}

%Größe der Ränder setzen
\geometry{a4paper,left=3cm, right=3cm, top=3cm, bottom=3cm}

%Kopf- und Fußzeile
\pagestyle {fancy}
\fancyhead[L]{Tutor: \TUTOR}
\fancyhead[C]{\COURSE}
\fancyhead[R]{\today}

\fancyfoot[L]{}
\fancyfoot[C]{}
\fancyfoot[R]{Seite \thepage /\pageref*{LastPage}}

\newcounter{aufgabe}

%Formatierung der Überschrift, hier nichts ändern
\def\header#1#2{
  \begin{center}
    {\Large Exercise Sheet #1}\\
    {(Deadline #2)}
  \end{center}
}

\newcommand{\nextAufgabe}{
  \section*{Aufgabe \theaufgabe}
  \stepcounter{aufgabe}
}
\newcommand{\circled}[2]{
  \tikz[baseline=(char.base)]{
  \node[shape=circle,draw,color=#2,inner sep=2pt] (char) {#1};}
}

\newcommand{\rc}[1]{
  \circled{#1}{red}
}
\newcommand{\gc}[1]{
  \circled{#1}{green}
}
\newcommand{\bc}[1]{
  \circled{#1}{blue}
}

\newcommand{\R}{
  \mathbb{R}
}
\newcommand{\N}{
  \mathbb{N}
}
\newcommand{\Z}{
  \mathbb{Z}
}
\newcommand{\Q}{
  \mathbb{Q}
}
\newcommand{\C}{
  \mathbb{C}
}

\newcounter{spalte}
\newcounter{zeile}

\newcommand{\spalte}[3]{
\left(\begin{array}{c}
  #1   \\
  #2   \\
  #3   \\
\end{array}\right)
}
\newcommand{\spaltef}[4]{
\left(\begin{array}{c}
  #1   \\
  #2   \\
  #3   \\
  #4   \\
\end{array}\right)
}
\newcommand{\kopf}[4]{
  
    idx & 
    %x werte
      \setsepchar{ }
      \forloop{spalte}{0}{\value{spalte} < #1}%
        {
        \readlist\arg{#3}
        \arg[\fpeval{\thespalte+1}] &
        } 
      %funktionen
      \setsepchar{ }
      \forloop{spalte}{0}{\value{spalte} < #2}%
        {
        \readlist\arg{#4}
        \arg[\fpeval{\thespalte+1}] \ifthenelse{\fpeval{#2-1}=\thespalte}{}{&}
        }
        \\\hline 
}

\begin{document}

\begin{tabularx}{\linewidth}{m{0.5 \linewidth} X}
  \begin{minipage}{\linewidth}
    \STUDENTA\\
    \STUDENTB\\
  \end{minipage} &
\end{tabularx}
\setcounter{aufgabe}{\AUFGABENSTART}%
\header{Nr. \NUMBER}{\DEADLINE}

% ----------------------- TODO ---------------------------

\begin{center}
  \begin{tabular}{|c|cc|cccc|}
    \hline
    Task      & 1.1         & 1.2        & 2.1       & 2.2       & 2.3     & 2.4  \\
    \hline
    Completed & $\boxtimes$ & $\boxtimes$  & $\boxtimes$ & $\boxtimes$ & $\boxtimes$ & $\boxtimes$ \\
    \hline
    Feedback  & $\square$   & $\square$    & $\square$ & $\square$ & $\square$ & $\square$ \\
    \hline
  \end{tabular}
\end{center}


\section*{Exercise 1 -  Cyclomatic Complexity}

\subsection*{Exercise 1.1}

\subsection*{Exercise 1.2}
The alternative choice of the CFG construction would not alter the value of the cyclomatic Complexity metric since:
\begin{center}
  $V(G) = |E| - |V| + p = 10 - 9 + 2 = 3$
\end{center}
which is the holds the same value as the original CFG. We can thereby conclude that the cyclomatic complexity metric 
is independent of wheather we chose one version or the other. For overview sake I would argue that represented version 
is way more readable and easier to understand.

\section*{Exercise 2 - Cost Estimation}

\end{document}
