% ----------------------- TODO ---------------------------
% Diese Daten müssen pro Blatt angepasst werden:
\newcommand{\NUMBER}{2}
\newcommand{\EXERCISES}{3}
\newcommand{\AUFGABENSTART}{1}
\newcommand{\DEADLINE}{- Freitag bis 12 Uhr}
% Diese Daten müssen einmalig pro Vorlesung angepasst werden:
\newcommand{\COURSE}{Stochastik fur Studierende der Informatik}
\newcommand{\TUTOR}{Timo Enger}
\newcommand{\STUDENTA}{Albert Ratschinski (5154309)}
% ----------------------- TODO ---------------------------

\documentclass[a4paper]{scrartcl}

\usepackage[utf8]{inputenc}
\usepackage[ngerman]{babel}
\usepackage{dsfont}
\usepackage{amsmath}
\usepackage{amssymb} % Added for square symbols
\usepackage{fancyhdr}
\usepackage{color}
\usepackage{graphicx}
\usepackage{lastpage}
\usepackage{listings}
\usepackage{tikz}
\usepackage{pdflscape}
\usepackage{subfigure}
\usepackage{float}
\usepackage{polynom}
\usepackage{hyperref}
\usepackage{tabularx}
\usepackage{forloop}
\usepackage{geometry}
\usepackage{fancybox}
\usepackage{stmaryrd}
\usepackage{bbold}
\usepackage{enumitem} % This package allows customization of lists

\usetikzlibrary{calc,arrows}
\usepackage{listings}
\lstset{
    basicstyle=\small\ttfamily,
    breaklines=true,
    numbers=left,
    numberstyle=\tiny,
    frame=tb,
    columns=fullflexible
}

%Größe der Ränder setzen
\geometry{a4paper,left=3cm, right=3cm, top=3cm, bottom=3cm}

%Kopf- und Fußzeile
\pagestyle {fancy}
\fancyhead[L]{Tutor: \TUTOR}
\fancyhead[C]{\COURSE}
\fancyhead[R]{\today}

\fancyfoot[L]{}
\fancyfoot[C]{}
\fancyfoot[R]{Seite \thepage /\pageref*{LastPage}}

\newcounter{aufgabe}

%Definition der Punktetabelle, hier nichts ändern

\newcommand{\punkteliste}[2]{%

  \begin{center}%
  \begin{tabular}{*{#2}{c|} c| c}

  Aufgabe: & 
  \forloop{aufgabe}{0}{\value{aufgabe} < #2}%
  {
    \fpeval{\value{aufgabe}+#1} &
  }   $\sum $\\
 \hline
 Punkte:
 \forloop{aufgabe}{-1}{\value{aufgabe} < #2}%
  {
    &
  } 
  \end{tabular}
  \end{center}
}

%Formatierung der Überschrift, hier nichts ändern
\def\header#1#2{
  \begin{center}
    {\Large Exercise Sheet #1}\\
    {(Deadline #2)}
  \end{center}
}

\newcommand{\nextAufgabe}{
  \section*{Aufgabe \theaufgabe}
  \stepcounter{aufgabe}
}
\newcommand{\circled}[2]{
  \tikz[baseline=(char.base)]{
  \node[shape=circle,draw,color=#2,inner sep=2pt] (char) {#1};}
}

\newcommand{\rc}[1]{
  \circled{#1}{red}
}
\newcommand{\gc}[1]{
  \circled{#1}{green}
}
\newcommand{\bc}[1]{
  \circled{#1}{blue}
}

\newcommand{\R}{
  \mathbb{R}
}
\newcommand{\N}{
  \mathbb{N}
}
\newcommand{\Z}{
  \mathbb{Z}
}
\newcommand{\Q}{
  \mathbb{Q}
}
\newcommand{\C}{
  \mathbb{C}
}

\newcounter{spalte}
\newcounter{zeile}

\newcommand{\spalte}[3]{
\left(\begin{array}{c}
  #1   \\
  #2   \\
  #3   \\
\end{array}\right)
}
\newcommand{\spaltef}[4]{
\left(\begin{array}{c}
  #1   \\
  #2   \\
  #3   \\
  #4   \\ 
\end{array}\right)
}
\newcommand{\kopf}[4]{
  
    idx & 
    %x werte
      \setsepchar{ }
      \forloop{spalte}{0}{\value{spalte} < #1}%
        {
        \readlist\arg{#3}
        \arg[\fpeval{\thespalte+1}] &
        } 
      %funktionen
      \setsepchar{ }
      \forloop{spalte}{0}{\value{spalte} < #2}%
        {
        \readlist\arg{#4}
        \arg[\fpeval{\thespalte+1}] \ifthenelse{\fpeval{#2-1}=\thespalte}{}{&}
        }
        \\\hline 
}

\begin{document}

\begin{tabularx}{\linewidth}{m{0.5 \linewidth} X}
    \begin{minipage}{\linewidth}
      \STUDENTA\\
    \end{minipage} & 
    \begin{minipage}{\linewidth}
      \punkteliste{\AUFGABENSTART}{\EXERCISES}
    \end{minipage}\\
  \end{tabularx}
  \setcounter{aufgabe}{\AUFGABENSTART}%
  \header{Nr. \NUMBER}{\DEADLINE}

% ----------------------- TODO ---------------------------
\section*{Aufgabe 1}
\subsection*{(a)}
Gegeben sind:
\begin{itemize}
    \item 52 Buchstaben (Groß- und Kleinschreibung)
    \item 10 Ziffern (0 bis 9)
\end{itemize}

Der Nutzername hat eine Länge von 5 Zeichen. Die Verteilung von Buchstaben und Ziffern kann folgende Formen annehmen:
\begin{itemize}
    \item 1 Buchstabe und 4 Ziffern
    \item 2 Buchstaben und 3 Ziffern
    \item 3 Buchstaben und 2 Ziffern
    \item 4 Buchstaben und 1 Ziffer
\end{itemize}

Da keine Ziffer vor einem Buchstaben stehen darf, kommen die Ziffern immer nach den Buchstaben. Die Anzahl der Möglichkeiten für jede Konfiguration lässt sich mit der Formel berechnen:

\[
\text{Gesamtzahl} = \sum_{k=1}^{4} \binom{5}{k} \cdot 52^k \cdot 10^{5-k}
\]

Dabei ist \( k \) die Anzahl der Buchstaben. Der Ausdruck \( \binom{5}{k} \) ist der Binomialkoeffizient, der die Anzahl der Möglichkeiten angibt, \( k \) Buchstabenpositionen aus 5 auszuwählen. \( 52^k \) ist die Anzahl der Möglichkeiten, \( k \) Buchstaben auszuwählen, und \( 10^{5-k} \) ist die Anzahl der Möglichkeiten, \( 5-k \) Ziffern auszuwählen.

Die einzelnen Terme lauten somit:
\begin{align*}
\text{Ein Buchstabe, vier Ziffern:} & \quad \binom{5}{1} \cdot 52^1 \cdot 10^4, \\
\text{Zwei Buchstaben, drei Ziffern:} & \quad \binom{5}{2} \cdot 52^2 \cdot 10^3, \\
\text{Drei Buchstaben, zwei Ziffern:} & \quad \binom{5}{3} \cdot 52^3 \cdot 10^2, \\
\text{Vier Buchstaben, eine Ziffer:} & \quad \binom{5}{4} \cdot 52^4 \cdot 10^1.
\end{align*}

Addieren wir diese Werte, erhalten wir die Gesamtzahl der möglichen Nutzernamen unter den gegebenen Einschränkungen.

\[
\text{Gesamtzahl} = \binom{5}{1} \cdot 52^1 \cdot 10^4 + \binom{5}{2} \cdot 52^2 \cdot 10^3 + \binom{5}{3} \cdot 52^3 \cdot 10^2 + \binom{5}{4} \cdot 52^4 \cdot 10^1 = 535.828.800 \text{ Möglichkeiten}
\]

\subsection*{(b)}
Gegeben sind: 
\begin{itemize}
    \item 52 Buchstaben (Groß- und Kleinschreibung)
    \item 10 Ziffern (0 bis 9)
    \item 8 Sonderzeichen
\end{itemize}
Die ersten drei Zeichen des Passworts müssen sich von denen des Nutzernamens unterscheiden. Daher gibt es genau $70 - 1$ Optionen für jedes der ersten drei Zeichen des Passworts, da sie nicht mit den entsprechenden Zeichen des Nutzernamens übereinstimmen dürfen.
Für die verbleibenden fünf Zeichen des Passworts gibt es keine solchen Einschränkungen. Daher gibt es für jedes dieser Zeichen $70$ mögliche Optionen.
Somit haben wir:

\[
  \text{Gesamtzahl} = 69 \cdot 69 \cdot 69 \cdot 70 \cdot 70 = 1.609.694.100 \text{ Möglichkeiten}
\]

\subsection*{(c)}
Die Gesamtzahl der Paare (Nutzername, Passwort) ist das Produkt der Anzahl der möglichen Nutzernamen und der Anzahl der möglichen Passwörter.
Somit:
\[
  \text{Gesamtzahl} = 535.828.800 \cdot 1.609.694.100 = 8,6252045797008e+17
\]

\section*{Aufgabe 2}
Der Grundraum $\Omega$ umfasst alle möglichen Positionen auf dem Schachbrett. Da es sich um ein 8x8-Schachbrett handelt und jeder Figur 64 Positionen einnehmen kann, gibt es:
\[
  |\Omega| = 64 \cdot 64 = 4096 \text{ mögliche Positionen}
\]
\subsection*{(a)}
Wir definieren die Menge A als die Menge der Positionen, die von einem Turm eingenommen werden können. Ein Turm kann sich in einer Zeile oder Spalte bewegen, daher gibt es 8 mögliche Positionen in einer Zeile und 8 mögliche Positionen in einer Spalte.
Da der Turm sich nur in einer Zeile oder Spalte bewegen kann, gibt es insgesamt 16 mögliche Positionen, die der Turm einnehmen kann. Somit:
\[
  A = \{(x,y) \in \Omega  | row(x) = row(y) \text{ oder } col(x) = col(y)\}
\]
Es folgt:
\[
  |A| = 8 \cdot 64 + 8 \cdot 64 - 64 = 960.
\]
\text{(-64 für die Position, in der beide Türme gleichzeitig in der gleichen Zeile oder Spalte sind.)}
und somit:
\[
  P(A) = \frac{|A|}{|\Omega|} = \frac{960}{4096} = \frac{15}{64}
\]
\subsection*{(b)}
Wir definieren die Menge B als die Menge der Positionen, die von einem Läufer eingenommen werden können. Ein Läufer kann sich entlang einer Diagonale bewegen, daher gibt es 8 mögliche Positionen entlang einer Diagonale.
\[
  S = \{(x,y) \in \Omega  | Farbe(x) = Frabe(y)\}
\]
und
\[
  B = \{(x,y) \in S  | diag(x) = diag(y)\}
\]
Es folgt:
\[
  |B| = 2 \cdot (\binom{8}{2} + 2 \cdot \binom{7}{2} + 2 \cdot \binom{6}{2} + 2 \cdot \binom{5}{2} + 2 \cdot \binom{4}{2} + 2 \cdot \binom{3}{2} + 2 \cdot \binom{2}{2}) = 280.
\]

\[
  P(B) = \frac{1}{2}\frac{|B|}{|\Omega|} = \frac{1}{2}\frac{280}{32*32} = \frac{}{}
\]
\end{document}
