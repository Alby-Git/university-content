% ----------------------- TODO ---------------------------
% Diese Daten müssen pro Blatt angepasst werden:
\newcommand{\NUMBER}{1}
\newcommand{\EXERCISES}{6}
\newcommand{\AUFGABENSTART}{1}
\newcommand{\DEADLINE}{- Freitag bis 14 Uhr}
% Diese Daten müssen einmalig pro Vorlesung angepasst werden:
\newcommand{\COURSE}{Mathematik II für Informatiker - SS 2024}
\newcommand{\TUTOR}{Clara Wasmer}
\newcommand{\STUDENTA}{Albert Ratschinski (5154309)}
% ----------------------- TODO ---------------------------

\documentclass[a4paper]{scrartcl}

\usepackage[utf8]{inputenc}
\usepackage[ngerman]{babel}
\usepackage{dsfont}
\usepackage{amsmath}
\usepackage{amssymb} % Added for square symbols
\usepackage{fancyhdr}
\usepackage{color}
\usepackage{graphicx}
\usepackage{lastpage}
\usepackage{listings}
\usepackage{tikz}
\usepackage{pdflscape}
\usepackage{subfigure}
\usepackage{float}
\usepackage{polynom}
\usepackage{hyperref}
\usepackage{tabularx}
\usepackage{forloop}
\usepackage{geometry}
\usepackage{fancybox}
\usepackage{stmaryrd}
\usepackage{bbold}
\usepackage{enumitem} % This package allows customization of lists

\usetikzlibrary{calc,arrows}
\usepackage{listings}
\lstset{
    basicstyle=\small\ttfamily,
    breaklines=true,
    numbers=left,
    numberstyle=\tiny,
    frame=tb,
    columns=fullflexible
}

%Größe der Ränder setzen
\geometry{a4paper,left=3cm, right=3cm, top=3cm, bottom=3cm}

%Kopf- und Fußzeile
\pagestyle {fancy}
\fancyhead[L]{Tutor: \TUTOR}
\fancyhead[C]{\COURSE}
\fancyhead[R]{\today}

\fancyfoot[L]{}
\fancyfoot[C]{}
\fancyfoot[R]{Seite \thepage /\pageref*{LastPage}}

\newcounter{aufgabe}

%Definition der Punktetabelle, hier nichts ändern

\newcommand{\punkteliste}[2]{%

  \begin{center}%
  \begin{tabular}{*{#2}{c|} c| c}

  Aufgabe: & 
  \forloop{aufgabe}{0}{\value{aufgabe} < #2}%
  {
    \fpeval{\value{aufgabe}+#1} &
  }   $\sum $\\
 \hline
 Punkte:
 \forloop{aufgabe}{-1}{\value{aufgabe} < #2}%
  {
    &
  } 
  \end{tabular}
  \end{center}
}

%Formatierung der Überschrift, hier nichts ändern
\def\header#1#2{
  \begin{center}
    {\Large Exercise Sheet #1}\\
    {(Deadline #2)}
  \end{center}
}

\newcommand{\nextAufgabe}{
  \section*{Aufgabe \theaufgabe}
  \stepcounter{aufgabe}
}
\newcommand{\circled}[2]{
  \tikz[baseline=(char.base)]{
  \node[shape=circle,draw,color=#2,inner sep=2pt] (char) {#1};}
}

\newcommand{\rc}[1]{
  \circled{#1}{red}
}
\newcommand{\gc}[1]{
  \circled{#1}{green}
}
\newcommand{\bc}[1]{
  \circled{#1}{blue}
}

\newcommand{\R}{
  \mathbb{R}
}
\newcommand{\N}{
  \mathbb{N}
}
\newcommand{\Z}{
  \mathbb{Z}
}
\newcommand{\Q}{
  \mathbb{Q}
}
\newcommand{\C}{
  \mathbb{C}
}

\newcounter{spalte}
\newcounter{zeile}

\newcommand{\spalte}[3]{
\left(\begin{array}{c}
  #1   \\
  #2   \\
  #3   \\
\end{array}\right)
}
\newcommand{\spaltef}[4]{
\left(\begin{array}{c}
  #1   \\
  #2   \\
  #3   \\
  #4   \\ 
\end{array}\right)
}
\newcommand{\kopf}[4]{
  
    idx & 
    %x werte
      \setsepchar{ }
      \forloop{spalte}{0}{\value{spalte} < #1}%
        {
        \readlist\arg{#3}
        \arg[\fpeval{\thespalte+1}] &
        } 
      %funktionen
      \setsepchar{ }
      \forloop{spalte}{0}{\value{spalte} < #2}%
        {
        \readlist\arg{#4}
        \arg[\fpeval{\thespalte+1}] \ifthenelse{\fpeval{#2-1}=\thespalte}{}{&}
        }
        \\\hline 
}

\begin{document}

\begin{tabularx}{\linewidth}{m{0.5 \linewidth} X}
    \begin{minipage}{\linewidth}
      \STUDENTA\\
    \end{minipage} & 
    \begin{minipage}{\linewidth}
      \punkteliste{\AUFGABENSTART}{\EXERCISES}
    \end{minipage}\\
  \end{tabularx}
  \setcounter{aufgabe}{\AUFGABENSTART}%
  \header{Nr. \NUMBER}{\DEADLINE}

% ----------------------- TODO ---------------------------
\section*{Aufgabe 1}

\[
(A|b)=
\begin{pmatrix}
  1 & 3 & 0 & | & 1 \\ 
  2 & 1 & \lambda & | & 1 \\
  1 & 5 & 4 & | & \beta
\end{pmatrix}
\]

Ich wende die Gauß-Elimination an und erhalte:

\[
\begin{array}{c@{\hspace{2em}}l}
\begin{pmatrix}
1 & 3 & 0 & | & 1 \\
2 & 1 & \lambda & | & 1 \\
1 & 5 & 4 & | & \beta
\end{pmatrix}
&
\begin{array}{l}
II - 2 \times I  \\
III + I \times - 1\\
\end{array}
\end{array}
\]

\[
\begin{array}{c@{\hspace{2em}}l}
\begin{pmatrix}
1 & 3 & 0 & | & 1 \\
0 & -5 & \lambda & | & -1 \\
0 & 2 & 4 & | & \beta - 1
\end{pmatrix}
&
\begin{array}{l}
III + \frac{2}{5} \times II
\end{array}
\end{array}
\]

\[
\begin{array}{c@{\hspace{2em}}l}
\begin{pmatrix}
1 & 3 & 0 & | & 1 \\
0 & -5 & \lambda & | & -1 \\
0 & 0 & \frac{2 \lambda + 20}{5} & | & \frac{5 \beta - 7}{5}
\end{pmatrix}
&
\begin{array}{l}
III + \frac{2}{5} \times II
\end{array}
\end{array}
\]

\begin{itemize}
  \item Fall 1: $\lambda \neq -10$
  
  Wir erhalten $\varepsilon_3$ durch Rückwärtseinsetzen:
  \begin{align*}
    (\frac{2\lambda + 20}{5})\varepsilon_3 &= \frac{5\beta - 7}{5} \\
    \Leftrightarrow \varepsilon_3 &= \frac{5\beta - 7}{2\lambda + 20}
  \end{align*}

  Äquivalent dazu erhalten wir $\varepsilon_2$ durch Rückwärtseinsetzen:
  \begin{align*}
    -5\varepsilon_2 + \lambda \varepsilon_3 &= -1 \\
    \Leftrightarrow -5 \varepsilon_2 &= 1 - \lambda \cdot \frac{5\beta - 7}{2\lambda + 20} \\
    \Leftrightarrow -5 \varepsilon_2 &=  1 - \frac{5\beta \lambda - 7 \lambda}{2\lambda + 20} \\
    \Leftrightarrow \varepsilon_2 &= \frac{-5 \lambda - 5 \lambda \beta - 20}{5(2\lambda + 20)} \\
    \Leftrightarrow \varepsilon_2 &= \frac{-\lambda + \lambda \beta + 4}{2\lambda + 20}
  \end{align*}

  Äquivalent dazu erhalten wir $\varepsilon_1$ durch Rückwärtseinsetzen:
  \begin{align*}
    1 \varepsilon_1 + 3 \varepsilon_2 &= 1 \\
    \Leftrightarrow \varepsilon_1 &= 1 - 3 \cdot \frac{-\lambda + \lambda \beta + 4}{2\lambda + 20} \\
    \Leftrightarrow \varepsilon_1 &= 1 - \frac{-3\lambda + 3\lambda \beta + 12}{2\lambda + 20} \\
    \Leftrightarrow \varepsilon_1 &= \frac{-\lambda +3 \lambda \beta + 8}{2\lambda + 20} \\
  \end{align*}

  Somit ergibt sich die Lösungsmenge:
  \[
  L_{\alpha, \beta}(G) = \left\{ \left( \frac{-\lambda +3 \lambda \beta + 8}{2\lambda + 20}, \frac{-\lambda + \lambda \beta + 4}{2\lambda + 20}, \frac{5\beta - 7}{2\lambda + 20} \right) \mid \lambda \neq -10 \right\}
  \]

  \item Fall 2: $\lambda = -10$
  
  Für den Fall $\frac{5\beta}{7} \neq 0$ gibt es keine Lösung:
  \[
  \begin{pmatrix}
    1 & 3 & 0 & | & 1 \\
    0 & -5 & \lambda & | & -1 \\
    0 & 0 & 0 & | & \frac{5 \beta - 7}{5}
  \end{pmatrix}
  \]

  Für den Fall $\frac{5\beta}{7} = 0$ gibt es unendlich viele Lösungen: \\\\
  Wir wählen $\varepsilon_3$ als freie Variable und erhalten $\varepsilon_2$ durch Rückwärtseinsetzen:

  \begin{align*}
    -5 \varepsilon_2 + \lambda \varepsilon_3 &= -1 \\
    \Leftrightarrow \varepsilon_2 &= \frac{-1 - \lambda \varepsilon_3}{-5} \\
    \Leftrightarrow \varepsilon_2 &= \frac{1}{5} + \frac{\lambda}{5} * \varepsilon_3
  \end{align*}

  Äquivalent dazu erhalten wir $\varepsilon_1$ durch Rückwärtseinsetzen:
  \begin{align*}
    \varepsilon_1 + 3 \varepsilon_2 &= 1 \\
    \Leftrightarrow \varepsilon_1 &= 1 - 3 \left( \frac{1}{5} + \frac{\lambda}{5} * \varepsilon_3 \right) \\
    \Leftrightarrow \varepsilon_1 &= 1 - \frac{3}{5} - \frac{3\lambda}{5} * \varepsilon_3 \\
    \Leftrightarrow \varepsilon_1 &= \frac{2}{5} - \frac{3\lambda}{5} * \varepsilon_3
  \end{align*}

  Somit ergibt sich die Lösungsmenge:
  \[
  L_{\alpha, \beta}(G) = \left\{ \left( \frac{2}{5} - \frac{3\lambda}{5} * \varepsilon_3, \frac{1}{5} + \frac{\lambda}{5} * \varepsilon_3, \varepsilon_3 \right) \mid \lambda = -10, \varepsilon_3 \in \mathbb{R}  \right\}
  \]
\end{itemize}

\subsection*{Aufgabe 2}

\begin{itemize}
  \item \textbf{Zu zeigen:} Die Relation $\sim$ definiert durch $a \sim b :\Leftrightarrow n \text{ teilt } a - b$ ist eine Äquivalenzrelation auf $\mathbb{Z}$.

  \item \textbf{Beweis:}
  \begin{enumerate}
    \item \textbf{Reflexivität:} Sei $a \in \mathbb{Z}$. Dann gilt $n \mid a - a = 0$, also $a \sim a$.
    
    \item \textbf{Symmetrie:} Seien $a, b \in \mathbb{Z}$ mit $a \sim b$. Das bedeutet, dass $n \mid a - b$. Also gibt es ein $k \in \mathbb{Z}$, sodass $a - b = kn$. Dann ist $b - a = -(a - b) = -kn$, was bedeutet, dass $n \mid b -
    a$. Also ist $b \sim a$.

    \item \textbf{Transitivität:} Seien $a, b, c \in \mathbb{Z}$ mit $a \sim b$ und $b \sim c$. Das bedeutet, dass $n \mid a - b$ und $n \mid b - c$. Also gibt es $k, l \in \mathbb{Z}$, sodass $a - b = kn
    $ und $b - c = ln$. Dann ist $(a - c) = (a - b) + (b - c) = kn + ln = (k + l)n$, also $n \mid (a - c)$. Also ist $a \sim c$.
  \end{enumerate} 

  \item Da die Relation $\sim$ reflexiv, symmetrisch und transitiv ist, ist sie eine Äquivalenzrelation auf $\mathbb{Z}$.
  
  \item Äquivalenzklasse von $a \in \mathbb{Z}$ bezüglich $\sim$: $\{a + kn \mid k \in \mathbb{Z}\}$.
  \item Repräsentantensystem für die Äquivalenzklassen: $\{0, 1, 2, ..., n-1\}$.
\end{itemize}


\end{document}
